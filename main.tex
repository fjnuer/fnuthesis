\documentclass[master]{fnuthesis}

\xuexiaobianhao{10394}					% 学校编号
\xuehao{}								% 学号
\tushufenleihao{}						% 图书分类号
\miji{}									% 密级
\zhongwentiming{*****(中文题名)}		% 中文题目
\yingwentiming{****(英文题名)}			% 英文题目
\zuozhexingming{*****(作者姓名)}		% 作者姓名
\peiyangfangshi{全日制}					% 培养方式
\xueweileibie{专业学位}					% 学位类别
\xuekeleibie{xxx}							% 学科类别
\xuekezhuanye{}							% 学科专业
\yanjiufangxiang{}						% 研究方向
\zhidaojiaoshi{}						% 指导教师
\xueweijibie{硕士}						% 申请学位类别
\tijiaoriqi{\qquad 年\quad 月\quad 日}	% 论文提交日期
\pingyueren{}							% 论文评阅人
\dabianriqi{\qquad 年\quad 月\quad 日}	% 论文答辩日期
\weiyuanhuizhuxi{}						% 答辩委员会主席
\shouyudanwei{福建师范大学}				% 学位授予单位
\shouyuriqi{\qquad 年\quad 月\quad 日}	% 学位授予日期
\riqi{\qquad 年\quad 月}					% 日期

\addbibresource{references.bib}

\begin{document}

\frontmatter

\chapter{中文摘要}

正文(学位论文第一页为300-500汉字的中文摘要。中文摘要应说明论文的目的、研究方法、成果和结论。在本页的最下方另起一行,注明本文的关键词3-5个(包括标点符号总字数不超过15个汉字,关键词之间用“,”分开)。(小四号宋体,固定值18磅行距)

\vfill

关键词:

\chapter{Abstract}

学位论文第二页为英文摘要,其内容与中文摘要相同。在本页的最下方另起一行,注明本文的英文关键词3-5个。(小4号Times New Roman字体,固定值18磅行距)

\vfill

Keywords:

\chapter{中文文摘}

正文:中文文摘应不少于2000个汉字。中文文摘是学位论文的缩影,应尽可能保留原论文的基本信息,突出论文的新成果和新见解。摘要应尽量浅显明了、通俗易懂,少用公式字母,语言力求精炼、准确。(小四号宋体,固定值18磅行距)

\tableofcontents

\chapter{绪论}

内容为本研究领域的国内现状分析(如有国外现状分析更佳),本论文所要解决的问题,该研究工作对我国基础教育事业发展、改革与管理和科技进步等方面的理论意义与实践价值。(小四号宋体,固定值18磅行距)

\mainmatter

\chapter{xxxxxxxxx}

\section{xxxxxxxxx}

正文内容(宋体小四号,固定值18磅行距),脚注\footnote{脚注说明文字},脚注\footnote{脚注说明文字2}

\chapter{xxxxxxxxx}

\section{xxxxxxxxx}

\chapter{xxxxxxxxx}

\section{xxxxxxxxx}

\chapter{xxxxxxxxx}

\section{xxxxxxxxx}

\chapter{xxxxxxxxx}

\section{xxxxxxxxx}

\chapter{结论}

论文结论要求明确、精炼、完整、准确,认真阐述自己创造性成果或新见解在本领域的意义。应严格区分本人的研究成果与导师或其他人的科研成果的界限。(宋体小四号,固定值18磅行距)

\backmatter

\chapter{附录1}

\chapter{附录2}

\printbibliography[heading=bibnumbered]

\chapter{攻读学位期间承担的科研任务与主要成果}

\chapter{致谢}

\chapter{索引}

\chapter{个人简历}

\end{document}
